\section{Mein Erste C Programm}
Dieses Buch (Kurz) handelt sich um eine programmier Sprache. Die programmier Sprache ist ein Werkzeug, mit dem wir unser
Gedanken zum Computer versenden können. Zum beispiel wir möchten $n$ Zahlen sortieren. Wir wissen im Kopf,
wie wir es machen werden. Wir wissen der Algorithmus der Sortierungs. Die einfachste ist der Einfügesortieren.
Wir fassen zusammen in
der unteren Auflistung:
\begin{lstlisting}{Einf\"ugesortieren}
1; Wir haben zwei Listen :
     sortierte, unsortierte
2; Am Anfang die sortierte Liste besteht aus dem ersten Zahl
3; Alle andere Zahlen gehoren zum unsortierten Liste
4; Wir machen eine Schleife uber allen Element der unsortierten Liste
5; Fur jeder Elemente in der ursortierten Liste wir suchen
     fur die angemessene Position in der sortierten Liste.
6; Wir ziehen die Element um zu der sortierten Liste.
\end{lstlisting}
Das ist alles, aber das wird der Computer nicht verstehen. Wir müssen das übersetzten um der Computer
verstehen zu können. Jede sprache hat eigene Sprachkonstrukte um diese Übersetztung zu erledigen.

\subsection{Speichern}
Wir handeln Zahlen im Computer mithilfe der Speichern. Die Größe der Speicher hängt von der Art von Zahlen
ab. Zum Beispiel wir können Ganze Zahlen auf wenigen Platz speichern, als reelle Zahlen.
Als Speicher man kann Registers, Memory oder Festpaletten verwenden.

Alle speichern bestehen aus elementare Speicherzellen (mann nennt sie es Byte-s). Mann nennt diese zellen Elementare, obwohl es 8
Bauteile (mann nennt es  bit) hat. Mann kann sagen elementare Speicherzelle auch, weil einigen
bit kann nicht verändert werden. Ein Bit kann geladen or entladen werden, dabei hat es zwei zustände:
0 (nein), 1 (ja). Auf diese Weise in einem Speicherzelle kann man 256 vershiedenen Zustände (verschiedenen Zahlen)
speichern. Zu speichern mehreren Zustände es gibt verschiedene Möglichkeiten:
\begin{itemize}
\item Byte:  1Byte $2^{8 }$ zustände
\item Word:  2Byte $2^{16}$ zustände
\item Dword: 4Byte $2^{32}$ zustände
\item Qword: 8Byte $2^{64}$ zustände
\end{itemize}
Zum beispiel, wenn du schreibst deine Code in einem txt File, für alle geschriebene character nutzt man 1 Byte
speicher auf das Festpallette. Man speichers das ASCII charachter der Buchstabe.

Die Speichers haben zwei wichtigen Eigenschaften:
\begin{itemize}
\item Ihre Grösse
\item Zeit zu erreichen eine Zelle
\end{itemize}
Diese Eigenschaften zusammen bestimmen die Speichern. Zum Beispiel die Memory ist umgefähr 10.000 mal
schneller zu erreichen als die  Festpalletten, aber 50 mal langsamer zu erreichen als die Registers.
Aber die Registers besteht aus wenigen Kilobytes, das Memory aus wenigen Gbytes, und das Festpallette
aus wenigen TBytes. Glücklicherweise wir müssen nicht genau wissen, wie die verkehr zwischen haupt memory
und die Register behandeln wird, die Compiler macht diese Aufgabe.

\subsection{Der Körper des C code für Einfügesortieren}
Die Computer programme besteht aus Variablen, mit dem wir operationen machen können und Funkcionen. In der letzten Punkt
wir haben gesehen, wie kann man Zahlen im Computer speichern. Hier wir werden einführen, wie mann Variablen
deklarieren, Wert geben oder auf dem Monitor ausdrücken kann.

\subsubsection{Ausdrücken}
\begin{lstlisting}{Erste C programm}
#include<stdio.h>
int main(int argc, char *argv[]){
   printf("Hello world\n");
}
\label{typ_1}
\end{lstlisting}

Zuerst werden wir nur eine Nachricht auf dem Monitor Zeigen: "Hello World". Du siehst die Code oben. Es hat nur 4 Reihe, und
macht die complex Aufgabe: drücken auf Monitor. Jede C programm hat zwei Teile. In den ersten Teil du sagst dem Computer, welchen
Funkcionen die anderen geschrieben haben, willst du nutzten.  Das Funktion, ist eine Programmteil der macht ein Ausgang aus dem Eingang.
Hier in der dritten Reihe wir nutzten die printf Funkcion, was zeigt ihrer Argumente auf dem Monitor. Der Compiler muss wissen
wieviele and welche parameter jede Funktion haben kann. Dieses Information ist zusammelt von der sogennanten ''header`` Files.
Du siehst das in der ersten Reihe. Diese Reihe sagt dem Compiler zu beilagen das Inhalt des File "stdio.h". In diesem File
findet mann die deklaration des printf funktion. Mehr werden wir sagen über printf Funkcion gleich, aber hier wir also
haben unseren ersten Funkcion geschrieben. Das Name des Funkcions ist main. Jede C code muss ein Funkcio mit dem Namen main
haben. Die Ausführung des Programms beginnt mit diesem Funkcion. Es hat zwei Eingangsparameter:
\begin{enumerate}
\item Variable argc. Seines Wert ist gleich der Anzahl der Parameter.
\item Variable argv. Es ethält die Parameter
\end{enumerate}
Das Funkcion Main also hat Ausgangs Parameter, der Null ist, wenn alles war gut, und unsere Program hat erfolgreich
beendet.
