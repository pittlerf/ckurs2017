\documentclass{article}[12pt]
\usepackage[utf8]{inputenc}
\usepackage[T1]{fontenc}
\usepackage[ngerman]{babel}
\usepackage[dvipsnames]{xcolor}
\usepackage{lipsum}

\usepackage{amsfonts}
\usepackage[intlimits]{amsmath}
\usepackage{cite}
\usepackage{epsfig}

\usepackage[usenames,dvipsnames]{pstricks}
\usepackage{pstricks-add}
\usepackage{epsfig}
\usepackage{pst-grad} % For gradients
\usepackage{pst-plot} % For axes

\addtolength{\hoffset}{-1.5cm}
\addtolength{\textwidth}{3cm}
\usepackage{listings}
\usepackage{color}
\definecolor{mygreen}{rgb}{0,0.6,0}
\definecolor{mygray}{rgb}{0.5,0.5,0.5}
\definecolor{mymauve}{rgb}{0.58,0,0.82}
\PassOptionsToPackage{svgnames}{xcolor}
\usepackage{tcolorbox}
\usepackage{lipsum}
\tcbuselibrary{skins,breakable}
\usetikzlibrary{shadings,shadows}
\usepackage{siunitx}
\lstset{ %
  backgroundcolor=\color{white},   % choose the background color; you must add \usepackage{color} or \usepackage{xcolor}; should come as last argument
  basicstyle=\footnotesize,        % the size of the fonts that are used for the code
  breakatwhitespace=false,         % sets if automatic breaks should only happen at whitespace
  breaklines=true,                 % sets automatic line breaking
  captionpos=b,                    % sets the caption-position to bottom
  commentstyle=\color{mygreen},    % comment style
  deletekeywords={...},            % if you want to delete keywords from the given language
  escapeinside={\%*}{*)},          % if you want to add LaTeX within your code
  extendedchars=true,              % lets you use non-ASCII characters; for 8-bits encodings only, does not work with UTF-8
  frame=single,                    % adds a frame around the code
  keepspaces=true,                 % keeps spaces in text, useful for keeping indentation of code (possibly needs columns=flexible)
  keywordstyle=\color{blue},       % keyword style
  language=C,                      % the language of the code
  morekeywords={*,...},            % if you want to add more keywords to the set
  numbers=left,                    % where to put the line-numbers; possible values are (none, left, right)
  numbersep=5pt,                   % how far the line-numbers are from the code
  numberstyle=\tiny\color{mygray}, % the style that is used for the line-numbers
  rulecolor=\color{black},         % if not set, the frame-color may be changed on line-breaks within not-black text (e.g. comments (green here))
  showspaces=false,                % show spaces everywhere adding particular underscores; it overrides 'showstringspaces'
  showstringspaces=false,          % underline spaces within strings only
  showtabs=false,                  % show tabs within strings adding particular underscores
  stepnumber=1,                    % the step between two line-numbers. If it's 1, each line will be numbered
  stringstyle=\color{mymauve},     % string literal style
  tabsize=2,                       % sets default tabsize to 2 spaces
  title=\lstname                   % show the filename of files included with \lstinputlisting; also try caption instead of title
}

\usepackage{amssymb}

\newenvironment{myexampleblock}[1]{%
    \tcolorbox[beamer,%
    noparskip,breakable,
    colback=White,colframe=ForestGreen,%
    colbacklower=LimeGreen!75!White,%
    title=#1]}%
    {\endtcolorbox}

\newenvironment{myalertblock}[1]{%
    \tcolorbox[beamer,%
    noparskip,breakable,
    colback=White,colframe=Bittersweet,%
    colbacklower=Peach!75!White,%
    title=#1]}%
    {\endtcolorbox}

\newenvironment{myblock}[1]{%
    \tcolorbox[beamer,%
    noparskip,breakable,
    colback=White,colframe=RoyalBlue,%
    colbacklower=TealBlue!75!White,%
    title=#1]}%
    {\endtcolorbox}

\newenvironment{myexampleprogram}[1]{%
    \tcolorbox[beamer,%
    noparskip,breakable,
    colback=White,colframe=Goldenrod,%
    colbacklower=Yellow!75!White,%
    title=#1]}%
    {\endtcolorbox}
%--------
%\usepackage[magyar]{babel}
\title{Conway Spiel des Lebens}
\begin{document}
\maketitle
Wir haben ein Universum, dass durch eine zweidimensionale Gitter dargestellt ist. Jede Gitterzelle kann 
zwei Zustand einnehmen:
\begin{itemize}
\item Lebendig
\item Tot
\end{itemize}
In dem Spiel wir entwickeln dieses Universum und folgen die Zuständen der Gitterzellen.
Wir haben die folgenden Prinipen für die Entwicklung:
\begin{enumerate}
\item Wenn eine lebendige Zelle zu viel lebendige Nachbarn hat, sie wird getötet. Überbevölkerung. Beispielweise, es gibt keine
Platz mehr für Sie auf dem Markt.
\item Wenn eine lebendige Zelle zu wenig lebendige Nachbarn hat, sie wird getötet. Unterbevölkerung. Beispielweise, es gibt keine Markten, 
zu denen Sie ihre Produkten verkaufen kann.
\item Wenn eine Tote Zelle genug lebendige Nachbarn hat, sie wird wiederbelebt. Beispielweise, es lohnt sich zu wieder öffnen ein Geschäft, wenn 
es genug Käufer gibt. 
\end{enumerate}
In der Entwicklung wir gehen durch die ganze Gitter und ändern die Zuständen der Zellen nach den folgenden Regeln;
\begin{enumerate}
\item Eine lebendige Zelle stirbt, wenn sie weniger als zwei lebendige Nachbarzellen hat.
\item Eine lebendige Zelle mit zwei oder drei lebendigen Nachbarn lebt weiter.
\item Eine lebendige Zelle mit mehr als drei lebenden Nachbarzellen stirbt im nächsten Zeitschritt.
\item Eine tote Zelle wird wiederbelebt, wenn sie genau drei lebende Nachbarzellen hat.
\end{enumerate}
In dieser Aufgabe wir werden die obenen Regeln implementieren. In dem Spiel am Anfang haben wir ein zweidimensionale Gitter. In jedem Knoten wir haben
ein system der zwei verschiedene Zustände haben kann:
\begin{equation}
Zelle(x,y) = \left\{\begin{array}{rc} 1 & \mathrm{Zelle~ist~Lebende} \\0  & \mathrm{Zelle~ist~Tot} \\
\end{array}\right. (x\in[0\cdots L-1], y\in[0,\cdots L-1])
\end{equation}
Im Computer die zweidimensionale Universe ist als ein zweidimensionale Array
dargestellt. In diesem Array werden wir die zustände der Zellen speichern. Die Größe dieses Arrays ist von der Anzahl der Zellen bestimmt. 
Wir haben schon nicht über zweidimensionale Arrays gesprochen, trotzdem können wir diese Aufgabe erledigen. Jede zweidimensionale Array
kann durch die Folgenden zu einem eindimensional Array transformiert werden:
\begin{equation}
array2d(x,y)= array1d(x+L*y)
\end{equation}
Hier L ist die Größe des Gitters. In der Entwicklung die Regeln müssen auf jede Gitterzelle im Simulationsbereich angewendet werden.
Wir haben in sowohl in $x$ als auxh in $y$ Richtungen periodische Randbedingungen:
\begin{equation}
Zelle(x,y)= Zelle(x+L,y) (x\in[0\cdots L-1], y\in[0,\cdots L-1])
\end{equation}
\begin{equation}
Zelle(x,y)= Zelle(x, y+L) (x\in[0\cdots L-1], y\in[0,\cdots L-1])
\end{equation}
Eine Zelle ($(x,y)$) hat acht Nachbarn:
\begin{itemize}
\item  $Zelle(x-1,y)$
\item  $Zelle(x-1,y-1)$
\item  $Zelle(x-1,y+1)$
\item  $Zelle(x,y-1)$
\item  $Zelle(x,y+1)$
\item  $Zelle(x+1,y)$
\item  $Zelle(x+1,y-1)$
\item  $Zelle(x+1,y+1)$
\end{itemize}
Wir müssen drei verschieden Funktionen implementieren:
\begin{itemize}
\item Entwicklung: Gibt ein neues Gitter zurück
\item Ausdrücken: Drückt das Gitter auf Monitor
\item Spiel: Initializiert und mach eine undendliche Schleife
\end{itemize}
Beim Entwicklung wir müssen achten, dass wir die Origninale Gitter nicht im Prozess von der Entwicklung ändern darf. Ansonst unser Ergebnis wird falsch sein.

Beim Ausdrücken wir können die Folgenden Steuerzeichen als Argument vom \texttt{printf()} verwenden:
\begin{enumerate}
\item \textbackslash$e[1;1H$\textbackslash$e[2J$: Löschen den Bildschirm
\item \textbackslash$033[H$: Positionierung des Kurzors auf die links obene Ecke
\item \textbackslash$033[E$; Beginn ein neue Zeile
\item \textbackslash$033[107m$\textbackslash$033[34m$: Hintergrund im Weiß, zeichen im Blau
\item \textbackslash$033[m$: Zurück zum Standard
\end{enumerate}
Zur Initializierung wir können unsere Zufallszahlgenerator verwenden. Beispielweise wenn den
Zufallszahl kleiner als 0.5 ist, die Zelle wird Tot sein, anderfalls es wird Lebendig sein.
\end{document}
