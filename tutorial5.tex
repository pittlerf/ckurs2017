\documentclass{article}[12pt]
\usepackage[utf8]{inputenc}
\usepackage[T1]{fontenc}
\usepackage[ngerman]{babel}
\usepackage[dvipsnames]{xcolor}
\usepackage{lipsum}

\usepackage{amsfonts}
\usepackage[intlimits]{amsmath}
\usepackage{cite}
\usepackage{epsfig}

\usepackage[usenames,dvipsnames]{pstricks}
\usepackage{pstricks-add}
\usepackage{epsfig}
\usepackage{pst-grad} % For gradients
\usepackage{pst-plot} % For axes

\addtolength{\hoffset}{-1.5cm}
\addtolength{\textwidth}{3cm}
\usepackage{listings}
\usepackage{color}
\definecolor{mygreen}{rgb}{0,0.6,0}
\definecolor{mygray}{rgb}{0.5,0.5,0.5}
\definecolor{mymauve}{rgb}{0.58,0,0.82}
\PassOptionsToPackage{svgnames}{xcolor}
\usepackage{tcolorbox}
\usepackage{lipsum}
\tcbuselibrary{skins,breakable}
\usetikzlibrary{shadings,shadows}
\usepackage{siunitx}
\lstset{ %
  backgroundcolor=\color{white},   % choose the background color; you must add \usepackage{color} or \usepackage{xcolor}; should come as last argument
  basicstyle=\footnotesize,        % the size of the fonts that are used for the code
  breakatwhitespace=false,         % sets if automatic breaks should only happen at whitespace
  breaklines=true,                 % sets automatic line breaking
  captionpos=b,                    % sets the caption-position to bottom
  commentstyle=\color{mygreen},    % comment style
  deletekeywords={...},            % if you want to delete keywords from the given language
  escapeinside={\%*}{*)},          % if you want to add LaTeX within your code
  extendedchars=true,              % lets you use non-ASCII characters; for 8-bits encodings only, does not work with UTF-8
  frame=single,                    % adds a frame around the code
  keepspaces=true,                 % keeps spaces in text, useful for keeping indentation of code (possibly needs columns=flexible)
  keywordstyle=\color{blue},       % keyword style
  language=C,                      % the language of the code
  morekeywords={*,...},            % if you want to add more keywords to the set
  numbers=left,                    % where to put the line-numbers; possible values are (none, left, right)
  numbersep=5pt,                   % how far the line-numbers are from the code
  numberstyle=\tiny\color{mygray}, % the style that is used for the line-numbers
  rulecolor=\color{black},         % if not set, the frame-color may be changed on line-breaks within not-black text (e.g. comments (green here))
  showspaces=false,                % show spaces everywhere adding particular underscores; it overrides 'showstringspaces'
  showstringspaces=false,          % underline spaces within strings only
  showtabs=false,                  % show tabs within strings adding particular underscores
  stepnumber=1,                    % the step between two line-numbers. If it's 1, each line will be numbered
  stringstyle=\color{mymauve},     % string literal style
  tabsize=2,                       % sets default tabsize to 2 spaces
  title=\lstname                   % show the filename of files included with \lstinputlisting; also try caption instead of title
}

\usepackage{amssymb}

\newenvironment{myexampleblock}[1]{%
    \tcolorbox[beamer,%
    noparskip,breakable,
    colback=White,colframe=ForestGreen,%
    colbacklower=LimeGreen!75!White,%
    title=#1]}%
    {\endtcolorbox}

\newenvironment{myalertblock}[1]{%
    \tcolorbox[beamer,%
    noparskip,breakable,
    colback=White,colframe=Bittersweet,%
    colbacklower=Peach!75!White,%
    title=#1]}%
    {\endtcolorbox}

\newenvironment{myblock}[1]{%
    \tcolorbox[beamer,%
    noparskip,breakable,
    colback=White,colframe=RoyalBlue,%
    colbacklower=TealBlue!75!White,%
    title=#1]}%
    {\endtcolorbox}

\newenvironment{myexampleprogram}[1]{%
    \tcolorbox[beamer,%
    noparskip,breakable,
    colback=White,colframe=Goldenrod,%
    colbacklower=Yellow!75!White,%
    title=#1]}%
    {\endtcolorbox}
%--------
%\usepackage[magyar]{babel}
\title{Handeln Zeichenketten}
\begin{document}
\maketitle
\begin{itemize}
\item Funktion \texttt{strlen}:
\begin{lstlisting}
size_t strlen(const char *str);
\end{lstlisting}
Soll die länge vom \texttt{str} zurückgeben.
Achtung nur 
\begin{lstlisting}
return sizeof(str);
\end{lstlisting}
ist nicht richtig, dies wird die Nummer der allokierten Speichherzellen zurückgeben. Das ist gar nicht die länge
der aktuellen Zeichenkette. Die länge ist definiert als das Nummer der Zeichen bis die Endstring-Zeichen.
Beispielweise verwendung:
\begin{lstlisting}
#include<stdio.h>
size_t strlen( const char *src){
   int i=0;
/*Du sollst dies schreiben */
   return (size_t)i;
}
int main(){
   char source[100]="Hausaufgabe";
   int lange=strlen(source);
   printf("%d\n", lange);
}
\end{lstlisting}
Die Ausgabe soll:
\begin{lstlisting}
11
\end{lstlisting}
sein.
\item Funktion \texttt{strcpy}:
\begin{lstlisting}
char *strcpy(char *destination, char *source);
\end{lstlisting}
Copieren die Zeichenkette \texttt{source} zur \texttt{destination} (Die Endstring-Zeichen soll auch inbegriffen sein). 
Achtung, \texttt{destination} musst genug allokierten Speicherzellen haben um die \texttt{source} zu Speciher, und 
diese Speicherzellen sollen nicht überlappen mit den Speicherzellen von \texttt{source}.
Beispielweise verwendung:
\begin{lstlisting}
#include<stdio.h>
char *strcpy( char *dest, char *src){
/* Du sollst diesen Teil schreiben */   
   return dest;
}
int main(){
   char source[100]="Hausaufgabe";
   char destination[100];
   strcpy(destination, source);
   printf("%s\n", destination);
}
\end{lstlisting}
Die Ausgabe soll:
\begin{lstlisting}
Hausaufgabe
\end{lstlisting}
sein.
\item Funktion \texttt{strchr}
\begin{lstlisting}
char * strchr ( char * str, int character );
\end{lstlisting}
Such für das Zeichen \texttt{character} im Zeichenkette \texttt{str}. Es sucht für das erste Ereignis.
Wenn das Zeichen gefunden ist, die Rückgabewert is ein Zeiger auf dem ersten Ereignis, sonst das NULL 
Zeiger. Möglicher verwendung:
\begin{lstlisting}
/* strchr example */
#include <stdio.h>
char * strchr ( char * str, int character ){
/* Du sollst es schreiben*/
}
int main ()
{
  char str[] = "Das ist eine einfache Zeichenkette";
  char * pch;
  printf ("Wir suchen fuer 's' Zeichen im \"%s\"...\n",str);
  pch=strchr(str,'s');
  while (pch!=NULL)
  {
    printf ("gefunden am Postion %d\n",pch-str+1);
    pch=strchr(pch+1,'s');
  }
  return 0;
}
\end{lstlisting}
Die Ausgabe soll:
\begin{lstlisting}
Wir suchen fuer 's' Zeichen im "Das ist eine einfache Zeichenkette"...
gefunden am Postion 3
gefunden am Postion 6
\end{lstlisting}
sein.

\end{itemize}
\end{document}
