\section{Endprojekt}


In unserem ersten großen Programme (Einfügesortieren) wir hatten
zwei verschachtelten Schleifen. Eine Schleife geht durch die Elementen
und der andere versucht die richtigen Position zu finden. Diese
innere Schleife im durchschnitt also skaliert mit dem Nummer der Elementen.
Beispielweise für eine total zufällige Eingabe der sortierende Element ist im 
durchschnitt kleiner als die Hälfte der schon sortierten Elementen. Darum 
wir verbrauchen  auch in der inneren Schleife Computerzeit der mit 
dem Nummer von elementen ($n$) skaliert. Deswegen ist das Einfügesortieren
ein $n^2$ Algorithmus.

Eine mögliche entwicklung ist die Haldensortierung. In diesem Projekt werden wir 
diese Algorithmus kennenlernen und implementieren. Die grundsätztliche Datentyp ist 
das $heap$.

\begin{myexampleblock}{Definition: \texttt{Heap}}
Es ist ein ausgeglichener Binärebaum. Es gibt nur eine Bedingung für die Elementen:
jeden Element muss kleiner sein, als ihre Elter.
\end{myexampleblock} 

In einem Heap jede Element hat nur ein Elter, und kann maximal zwei Kinder
haben. Der Strichwort ausgeglichen bedeuted das die Höhe des Heaps mit
dem logarithmus von $n$ skaliert muss. Bespielweise wir zeigen ein Heap
von der $[10,12,6,5,9,13,1,7]$ Reihenfolge auf Abbildung \ref{heapexample}.
\begin{center}
\begin{figure}[!ht]
\begin{center}
% Generated with LaTeXDraw 2.0.8
% Mon Mar 06 15:43:52 CET 2017
% \usepackage[usenames,dvipsnames]{pstricks}
% \usepackage{epsfig}
% \usepackage{pst-grad} % For gradients
% \usepackage{pst-plot} % For axes
\scalebox{0.7} % Change this value to rescale the drawing.
{
\begin{pspicture}(-5,-4)(5,5)
\pscircle[linewidth=0.04,dimen=outer](0,4){.5}
\rput(0, 4){13}
\psline[linewidth=0.04cm](-0.5,3.33333)(-1,2.6666)
\psline[linewidth=0.04cm](-.7,2.9)(-1,2.6666)
\psline[linewidth=0.04cm](-.8,3.2)(-1,2.6666)

\pscircle[linewidth=0.04,dimen=outer](-1.5,2){.5}
\rput(-1.5, 2){10}
\psline[linewidth=0.04cm](0.5,3.33333)(1,2.6666)
\psline[linewidth=0.04cm](.7,2.9)(1,2.6666)
\psline[linewidth=0.04cm](.8,3.2)(1,2.6666)

\pscircle[linewidth=0.04,dimen=outer]( 1.5,2){.5}
\rput( 1.5, 2){12}

\psline[linewidth=0.04cm](-2.3,0.933333)(-1.9,1.466666)
\psline[linewidth=0.04cm](-2.35,1.2)(-2.3,0.933333)
\psline[linewidth=0.04cm](-2.0,1.0)(-2.3,0.933333)

\pscircle[linewidth=0.04,dimen=outer](-2.7,0.4){.5}
\psline[linewidth=0.04cm](-.8,0.933333)(-1.2,1.466666)
\psline[linewidth=0.04cm](-0.75,1.2)(-.8,0.933333)
\psline[linewidth=0.04cm](-1.1,1.0)(-.8,0.933333)

\rput(-2.7, 0.4){7}
\psline[linewidth=0.04cm](2.3,0.933333)(1.9,1.466666)
\psline[linewidth=0.04cm](2.35,1.2)(2.3,0.933333)
\psline[linewidth=0.04cm](2.0,1.0)(2.3,0.933333)

\psline[linewidth=0.04cm](.8,0.933333)(1.2,1.466666)
\psline[linewidth=0.04cm](0.75,1.2)(.8,0.933333)
\psline[linewidth=0.04cm](1.1,1.0)(.8,0.933333)


\pscircle[linewidth=0.04,dimen=outer](-0.8,0.4){.5}
\rput(-0.8, 0.4){9}
\pscircle[linewidth=0.04,dimen=outer](2.7,0.4){.5}
\rput(2.7, .4){1}
\pscircle[linewidth=0.04,dimen=outer](0.8,0.4){.5}
\rput(0.8, 0.4){6}

\psline[linewidth=0.04cm](-3.4,-0.533333333333333)(-3.1,-0.133333333333333)
\psline[linewidth=0.04cm](-3.,-.3)(-3.4,-0.533333333333333)
\psline[linewidth=0.04cm](-3.3,-.2)(-3.4,-0.533333333333333)

\pscircle[linewidth=0.04,dimen=outer](-3.75,-1){.5}
\rput(-3.75, -1){5}


\end{pspicture} 
}
\end{center}
\caption{Ein Heap von der $[10,12,6,5,9,13,1,7]$ Reihenfolge.\label{heapexample}}
\end{figure}
\end{center}
Wie du siehst, grundsätztlich es gibt keine Beziehung zwischen Elementen von der 
gleichen Höhe des Bäumes. Es gibt nur ein Regel: Das Kind muss kleiner sein, als ihres
Elter. Jetzt zeigen wir, wie es im Haldensortierung verwendet werden.
Er kann in folgen Schritten formuliert werden:
\begin{itemize}
\item Stellen wir ein $heap$ aus den Reihenfolge der Zahlen her
\item Nun verschieben wir jeweils das erste Element aus, und das wird mit dem letzten Element 
ersetzt.
\item Das $heap$ Eigenschaft muss wiederhergestellt werden.
\end{itemize}

Wir verwenden verketetten Liste für die Darstellung des $heap$-s. Jedes Element muss
ein Pointer für ihre Elter, und zwei Pointer für ihre rechten und linken Nachfolger haben.
\begin{lstlisting}
typedef struct heap {
   int val;
   struct heap *left;   /* Verweis auf linken Nachfolger  */
   struct heap *right;  /* Verweis auf rechten Nachfolger */
   struct heap *parent; /* Verweis auf den Vorgaenger */
} heap;
\end{lstlisting}
Der $parent$ Pointer ist ein NULL pointer für die Wurzel des Bäumes. Die Blätters, die keinen
Kinder haben, haben NULL als right und left Pointers.

Jetzt zeigen wir wie können wir ein Heap aus einem Reihenfolge von Zahlen herstellen. Zuerst
müssen wir den ersten freien Platz finden, dann fügen wir das Element hier ein. Danach es kann sein, dass 
das $heap$ Eigenschaft nicht wahr ist. Darum müssen wir das $heap$ Eigenschaft überprüfen und wenn
es notwendig ist, tauschen wir den aktuellen Wert mit dem Wert von ihrem Eltern bis das
$heap$ Eigenschaft wahr wird.
\begin{figure}[!ht]
\begin{center}
% Generated with LaTeXDraw 2.0.8
% Sun Mar 05 20:17:29 CET 2017
% \usepackage[usenames,dvipsnames]{pstricks}
% \usepackage{epsfig}
% \usepackage{pst-grad} % For gradients
% \usepackage{pst-plot} % For axes
\scalebox{0.7} % Change this value to rescale the drawing.
{
\begin{pspicture}(0,-4)(12.0,3)
\psframe[linewidth=0.04,dimen=outer](3,3)(0.0,2)
\psframe[linewidth=0.04,dimen=outer](11.0,-2.2)(8,-3.2)
\pscustom[linewidth=0.04]
{
\newpath
\moveto(3.1,2.5)
%\lineto(7.5,1.5)
\curveto(3.3,2.5)(3.5,2.4)(4.5,1.7)
}
\rput(4.7, 2.8){\LARGE queue\_put}
\psline[linewidth=0.04cm](4.55,1.9)(4.5,1.7)
\psline[linewidth=0.04cm](4.4,1.6)(4.5,1.7)


\pscustom[linewidth=0.04]
{
\newpath
\moveto(7.2,-1.6)
%\lineto(7.5,1.5)
\curveto(7.2,-1.6)(8,-1.2)(9.5,-2.1)
}
\rput(9, -1.){\LARGE queue\_rem}
\psline[linewidth=0.04cm](9.15,-2.2)(9.5,-2.1)
\psline[linewidth=0.04cm](9.2,-1.8)(9.5,-2.1)


\psframe[linewidth=0.04,dimen=outer](7,1.5)(4,0.5)
\psframe[linewidth=0.04,dimen=outer](7,0.2)(4,-0.8)
\psframe[linewidth=0.04,dimen=outer](7,-1.1)(4,-2.1)
\end{pspicture}
}
\caption{Die Warteschlange\label{warteschlange}}
\end{center}
\end{figure}

\begin{myexampleprogram}{ Programme: \texttt{Die Warteschlange}}	
\begin{lstlisting}
typedef struct queue {
   struct queue* next;
   int data;
} queue;
\end{lstlisting}
\begin{lstlisting}
queue * get_a_queue( int x ){
   queue *q=(queue *)malloc(sizeof(queue));
   q->data=x;
   q->next=NULL;
   return q;
}
\end{lstlisting}
\begin{lstlisting}
void queue_put( int x, queue ** first){
   if ((*first) == NULL){
       (*first)=get_a_queue(x);
       return;
   }
   queue *temp=get_a_queue(x);
   temp->next=(*first);
   (*first)=temp;
}
\end{lstlisting}
\begin{lstlisting}
int queue_rem( queue **first){
   int ret;
   queue *temp;
   if ((*first) == NULL){
     return -1;
   }
   if ((*first)->next == NULL){
     ret=(*first)->data;
     free((*first));
     (*first)=NULL;
     return ret;
   }
   temp=(*first);
   queue *prev;
   while ( temp->next != NULL){
      prev=temp;
      temp=temp->next;
   }
   prev->next=NULL;
   ret=temp->data;
   free(temp);
   return (ret);
}
\end{lstlisting}
\begin{lstlisting}
void free_queue( queue **first){
   if ((*first)== NULL)
     return;
   queue *temp=(*first);
   queue *temp2;
   for(;;){
     temp2=temp->next;
     free(temp);
     temp=temp2;
     if (temp == NULL)
       break;
   }
}
\end{lstlisting}
\begin{lstlisting}
void print_queue( queue *first ){
   queue *temp=(first);
   if (temp == NULL){
     printf("Wir haben keine Element in der Wertechlange\n");
     return;
   }
   for (;;){
     printf("%d\t", temp->data);
     temp=temp->next;
     if (temp == NULL)
       break;
   }
   printf("\n");
}
\end{lstlisting}
\begin{lstlisting}
int main(){
    queue *elso=NULL;
    int temp;
    queue_put(2, &elso);  print_queue(elso);
    queue_put(4, &elso);  print_queue(elso);
    queue_put(6, &elso);  print_queue(elso);
    temp=queue_rem( &elso ); printf("Ausgenommene Element: %d\n",temp); print_queue(elso);
    queue_put(8, &elso);  print_queue(elso);
    temp=queue_rem( &elso ); printf("Ausgenommene Element: %d\n",temp); print_queue(elso);
    free_queue( &elso );
}
\end{lstlisting}
\begin{lstlisting}{Output}
2	
4	2	
6	4	2	
Ausgenommene Element: 2
6	4	
8	6	4	
Ausgenommene Element: 4
8	6	
\end{lstlisting}
\end{myexampleprogram}
\begin{myexampleprogram}{Programme: \texttt{Haldensortierung}}
\begin{lstlisting}
typedef struct heap {
   int val;
   struct heap *left;
   struct heap *right;
   struct heap *parent;
} heap;
\end{lstlisting}
\begin{lstlisting}
heap *newheap( int x, heap *parent)
{
   heap *ret=(heap *)malloc(sizeof(heap));
   ret->left=NULL;
   ret->right=NULL;
   ret->parent=parent;
   ret->val=x;
   return ret;
}
\end{lstlisting}
\begin{lstlisting}
void heap_insert(heap** root, int x) {
   if(*root == NULL) {
     *root = newheap(x, NULL);
   }
   else {
     queue* q = newqueue(*root);
     heap* currentheap;

     while((currentheap=queue_rem(&q))!=NULL){

        if(currentheap->left == NULL) {
          currentheap->left  = newheap(x, currentheap);
          currentheap = currentheap->left;
        }
        else if(currentheap->right == NULL) {
          currentheap->right = newheap(x, currentheap);
          currentheap = currentheap->right;
        }
        else {
          queue_put(currentheap->left, &q);
          queue_put(currentheap->right,&q);
          continue;
        }
        while(currentheap->parent != NULL && currentheap->parent->val < currentheap->val) {
           int temp = currentheap->parent->val;
           currentheap->parent->val = currentheap->val;
           currentheap->val = temp;
           currentheap = currentheap->parent;
        }
        break;

     }
     free_queue(&q);
   }
   return;
}
\end{lstlisting}
\begin{lstlisting}
int heap_remove(heap** root) {
   int ret;
   if(*root != NULL) {
     queue* q = newqueue(*root);
     heap* previousheap;
     heap* currentheap;

     while ((currentheap=queue_rem(&q)) != NULL) {
       if (currentheap->left != NULL)
          queue_put(currentheap->left, &q);
       if (currentheap->right != NULL)
          queue_put(currentheap->right, &q);
       previousheap= currentheap;
     }
     currentheap = previousheap;
     free_queue(&q);

     if(currentheap->parent == NULL) {
       ret=currentheap->val;
       free(currentheap);
       *root = NULL;
       return ret;
     }
     else {
       ret=(*root)->val;
       (*root)->val = currentheap->val;
       currentheap = currentheap->parent;
       if(currentheap->right != NULL) {
          free(currentheap->right);
          currentheap->right = NULL;
       }
       else {
          free(currentheap->left);
          currentheap->left = NULL;
       }
       int a, b, c;
       currentheap = *root;
       while(1) {
         if(currentheap->left == NULL) {
           break;
         }
         else if(currentheap->right == NULL) {
           a = currentheap->val;
           b = currentheap->left->val;
           if(a < b) {
              currentheap->val = b;
              currentheap->left->val = a;
              currentheap = currentheap->left;
           }
           else {
             break;
           }
         }
         else {
           a = currentheap->val;
           b = currentheap->left->val;
           c = currentheap->right->val;
           if(a >= b && a >= c) {
             break;
           }
           else if(b > a && b >= c) {
             currentheap->left->val = a;
             currentheap->val = b;
             currentheap = currentheap->left;
           }
           else {
             currentheap->right->val = a;
             currentheap->val = c;
             currentheap = currentheap->right;
           }

         }
       }
       return ret;
     }
   }
   else
    return -1;
}
\end{lstlisting}
\end{myexampleprogram}
\subsubsection{Formattierte Eingabe und Ausgabe}
\begin{myexampleblock}{Function definition \texttt{printf}}
int printf(char * formattierung\_text, $\cdots$);
\begin{itemize}
\item Rückgabe Wert: Der Anzahl der ausgedrückten Zeichen
\item Parameters:
\begin{enumerate}
\item formattierung\_text: Eine Zeichenkette, beendet mit dem \'{}\\0\'{} Zeichen, specifiziert wie mann die Daten
audrücken will
\end{enumerate}
\end{itemize}
\end{myexampleblock}
\pagebreak
